\abstract{
The European Spallation Source will soon provide the most intense neutron beams for
multi-disciplinary science. Conveniently, it will also generate the
largest pulsed neutrino production rate suitable for the detection and study of Coherent
Elastic Neutrino-Nucleus Scattering (CE$\nu$NS), a process recently
measured for the first time at ORNL's Spallation Neutron Source. CE$\nu$NS holds the potential for significant advances in sensitivity to numerous aspects of particle and nuclear phenomenology, providing a new, long-sought tool in the search for physics beyond the Standard Model.\\

We propose to use innovative nuclear-recoil detector technologies, optimized to profit
from the order-of-magnitude increase in neutrino production that is expected from 
the ESS when compared to all other existing or planned spallation sources. These low-cost, compact, unintrusive devices profit from previous R\&D aimed at other particle physics applications. They can be deployed coincident with the start of the ESS user program, without any significant impact or deviation from the ESS neutron mandate. The combination of a state-of-the-art performance from these advanced detectors, together with the uniqueness of the ESS as a high-yield pulsed neutrino source, will return high-statistics, precision CE$\nu$NS
measurements at the limit of the sensitivity to new physics reachable through this novel neutrino interaction mechanism. \\

Exploiting the exciting new development and opportunity that CE$\nu$NS represents, the program described here will significantly expand the physics reach of the ESS to a new area, that of neutrino physics, at the expense of just a minimal investment of resources. 
}
